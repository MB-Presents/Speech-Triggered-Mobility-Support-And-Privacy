\section{Introduction}
Voice control is an interaction facility where technical devices can be controlled by human speech. The next song, the alarm clock or even order processes can be initiated with it. Experts predict a growing market for voice controls: The magazine "PR Newswire" estimates, that purchases over voice will increase twentyfold over the next four years \cite{prNewswire}. The magazine "Campaign" estimates that in the future the search in browsers will be replaced by the search via language \cite{Campaign}. The voice assistants include such voice control services and thus form the interface between users and applications.

The applications of a voice assistants are called apps and run on a platform in the cloud. Speech processing on the platform is sophisticated, as a user's speech input goes through highly complex speech processing sub-processes until an appropriate response can be generated for the user. Currently these platforms are offered by large cloud providers who have the financial resources and the know-how of each sub-process. Universities usually focus on a sub-process. voice assistants are offered by Amazon, Google, Microsoft or Baidu, with many features and good performance. However, there are privacy concerns, as the cloud providers privacy policies do not make it clear what happens to users' data in the cloud. Mobile devices such as smartphones and speakers send the voice input of a user for evaluation to the appropriate cloud provider. Data can be collected and possibly abused. The data usage is stated in the terms of use, but this gives a limited indication of the possible usage scenarios, such as the profiling of users.

For this reason, a survey was conducted with 110 participants. This includes the use of voice assistants, the relevance of data protection from the user's point of view and the financial readiness for more privacy. In doing so, users indicated applications where privacy is particularly important to them. The results are explained in chapter \ref{sec:motivaiton} and are the motivation for the development of the concept of a voice assistant with more privacy for the user. In chapter \ref{sec:konzept} a concept is presented by the user full control about their data. To fulfill this concept, an architecture has been developed for a voice assistant, which is discussed in chapter \ref{sec:architecure}. Afterwards, in chapter \ref{sec:umsetzung} technologies are presented, with which the architecture can be implemented. An assessment of the concept, the technologies and the implementation complete this article. \newline


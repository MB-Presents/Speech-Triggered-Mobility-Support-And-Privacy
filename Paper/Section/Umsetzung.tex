\section{Umsetzung}
In diesem Kapitel werden Technologien vorgestellt, mit denen die im letzten Kapitel vorgestellt Architektur umgesetzt werden kann. 

\subsection{Mobile App}
Die mobile App wird für verschiedenen mobile Plattformen wie Android, iOS, Windows Phone und als Webanwendung implementiert. Des Weiteren wäre auch die Integration der mobilen App in einen Lautsprecher eine Überlegung Wert. 
\begin{itemize}
	\item \textsl{Speech Recording:} Für das Aufnehmen der Eingabe eines Nutzers ist in den meisten Fällen keine zusätzliches Technologie notwendig. Fast alle mobilen Geräte besitzen bereits ein Mikrofon und die Geräte \ac{sdk}s stellen eine Schnittstelle bereit, mit der auf das Mikrofon zugegriffen werden kann.
	\item \textsl{Speech Playback:} Auch für das Abspielen eines Streams, kann der eingebaute Lautsprecher über die Geräte \ac{sdk} verwendet werden.
	\item \textsl{Hotword Detection:} Im folgenden werden Technologien vorgestellt, mit denen ein Signalworte auf dem mobilen Gerät erkannt werden können: 
	\begin{itemize}
		\item Snowboy Hotword Detection von Kitt.ai: Snowboy Hotword Detection ist ein Apache lizenziertes Software Projekt zur Erkennung von Hotwords. Das Hotword lässt sich frei bestimmen und das Software Projekt ist optimiert für eingebettete Systeme. Laut dem Hersteller Kitt.ai soll die Hotword Detection unter dem kleinsten Raspberry Pi (single-core 700MHz ARMv6) nur 10\% der CPU
		auslasten \cite{SnowboyHotwordDetection}.
		\item Sensory's TrulyHandsfreeTM: Sensory's TrulyHandsfreeTM ist eine Spracherkennung, die für die Erkennung von einzelnen Wörtern bzw. Kommandos optimiert wurde. Zudem zeigt sie sehr gute Ergebnisse in Umgebungen mit vielen Hintergrundger	äuschen auf. Das Vokabular, welches erkannt werden soll, kann durch Sensory's Grammatik Tool erstellt werden. Sensory's TrulyHandsfreeTM ist verfügbar für Android, iOS, QNX, Windows und Mikrocontroller \cite{TrulyHandsfreeTM}.
		\item Pocketsphinx von \acs{cmu} Sphinx: Sphinx ist ein Forschungsprojekt der \ac{cmu}, das sich mit Spracherkennung befasst. Sphinx basiert auf der Open-Source-Lizenz und kann somit frei verwendet werden, solange erkenntlich gemacht wird, dass es sich um eine Software von \ac{cmu} handelt. Pocketsphinx wurde ebenso wie Sensory's TrulyHandsfreeTM auf die Erkennung von einzelnen Wörter optimiert \cite{Pocketsphinx}.
	\end{itemize}
\end{itemize}

\subsection{Repository}
Das Repositorys kann durch einen Webserver umgesetzt werden. Dieser Webserver stellt einerseits verschiedene Laufzeitumgebungen und anderseits Apps für den Sprachassistenten zum Download bereit. Die Laufzeitumgebungen können in den folgenden Formaten angeboten werden:
\begin{itemize}
	\item \textsl{VMWare Image:} Das VMWare Image enthält bereits ein vorinstalliertes Betriebssystem sowie alle nötigen Pakete für den Sprachassistenten. Ein Nutzer muss dieses Image in seine VMWare Umgebung (VMWare vSphere, Workstation, Player) importieren und kleine Netzwerkkonfiguration vornehmen. Diese Umgebung lässt sich am einfachsten Nutzen und benötigt am wenigsten Konfigurationsaufwand. Zudem hat der Nutzer die Möglichkeit, die virtuelle Maschine auf einen anderen Server zu übertragen oder Snapshots zu erstellen. Snapshots sind Abbilder einer virtuellen Maschine, die einen bestimmten Zustand speichern. Dadurch kann bei einer Fehlkonfiguration einfach zum letzten funktionalen Abbild zurückgesprungen werden \cite{VMWare}.
	\item \textsl{Docker Image:} Das Docker Image ist eine Konfigurationsdatei für eine Container Umgebung. Diese Konfigurationsdatei beinhaltet alle zu installierenden Pakete und Konfiguration für den Sprachassistenten. Wird diese Konfigurationsdatei in einer Container Umgebung gestartet, werden die Pakete automatisch installiert und der Sprachassistent konfiguriert. Der Nutzer muss lediglich Netzwerkeinstellung vornehmen, bevor dieser den Sprachassistenten nutzen kann \cite{Docker}.
	\item \textsl{\acs{iso}-Image:} Das \acs{iso}-Image bietet dem Nutzer am meisten Flexibilität und Konfigurationsmöglichkeiten. Der Nutzer kann entscheiden, ob er die Laufzeitumgebung physikalisch oder virtualisiert installieren möchte. Des Weiteren ist dieses nicht an eine Virtualisierungssoftware wie VMWare gebunden und kann virtualisiert auf einer privaten oder öffentlichen Cloud installiert werden. Eine öffentlichen Cloud nimmt dem Nutzer Aufgaben wie Visualisierung, Backups, Ausfallsicherheit und Load Balancing ab, mögliche Anbieter sind \ac{aws} mit Amazon EC2 \cite{AWSAmazonEC2}, Microsoft Azure \cite{MicrosoftAzure} und IBM Bluemix \cite{IBMBluemix}. Das \acs{iso} Image beinhaltet auch alle nötigen Pakete für den Sprachassistenten. Der Nutzer wird bei der Installation des \acs{iso} Images durch ein Wizard geführt, bei dem alle Konfiguration vorgenommen werden.
\end{itemize}

\subsection{Laufzeitumgebung}
\subsubsection{Sprachverarbeitung}
In der Laufzeitumgebung werden Sprachverarbeitungsprozesse durchgeführt, es gibt zwei Möglichkeiten diese Prozesse durchzuführen, entweder lokal auf der Laufzeitumgebung oder durch die Nutzung von sprach-basierten Cloud Services. Im folgenden werden die Vor- bzw. Nachteile dieser Möglichkeiten sowie mögliche Technologien erläutert.
\begin{itemize}
	\item \textsl{Cloudbasierte Sprachverarbeitung:} Die Nutzung von sprach-basierte Cloud Services zur Sprachverarbeitung bietet die bestmögliche Performance. Die Sprachverarbeitung basiert i.d.R. auf einer \ac{ki}, welche sich durch Eingabedaten verbessert. Da Cloud Services von vielen Anwendungen genutzt werden, steht der \ac{ki} eine große Anzahl an Eingabedaten zur Verfügung. Dabei ergibt sich aber der Nachteil, dass die \ac{ki} Eingabedaten eines Nutzers, welche dessen Kontext beschreiben, zur Verbesserung der \ac{ki} genutzt werden und somit die Privatsphäre nicht optimal geschützt werden kann. Des Weiteren fallen bei der Nutzung von Cloud Services kosten an. Folgende Anbieter bieten sprach-basierte Cloud Services an:
	\begin{itemize}
		\item \ac{aws}: Amazon bietet zahlreiche sprach-basierte Cloud Services an. Amazon Comprehend ist ein Service zum \ac{nlu}, dabei können Einblicke in Zusammenhänge und Beziehungen eines Texte gewonnen werden \cite{AmazonComprehed}. Mit Amazon Translate können Texte, Webseiten und Anwendungen natürlich klingend und akkurat übersetzt werden \cite{AmazonTranslate}. Amazon Transcript kann Sprache zu Text und Amazon Polly Texte zu Sprache umwandeln \cite{AmazonTranscript} \cite{AmazonPolly}. Mit Amazon Lex können Konversationsschnittstellen für Anwendungen erzeugt werden. Dabei dient ein Chatbot als Konversationsschnittstelle und kann auf eine bestimmte Eingabe, die zugehörige Antwort liefern. Amazon Lex nutzt die gleichen Tieflerntechnologien als der Sprachassistent Alexa von Amazon \cite{AmazonLex}.
		\item Microsoft Azure: Microsoft Azure bietet Cognitive Services an, darunter Services zur Sprach zu Text, Text zu Sprache, Übersetzung von Texten und \ac{nlu}. Des Weiteren wird ein Service zur Sprechererkennung und zur Rechtschreibkorrektur angeboten \cite{MicrosoftAzureCognitiveServices}. Gerade die Sprechererkennung ist ein wichtiger Bestandteil für das in diesem Artikel vorgestellten Konzept eines Sprachassistenten, dieser Service kann zur Authentifizierung eines Nutzers genutzt werden. 
		\item IBM Watson: Auch IBM Watson bietet sprach-basierte Cloud Services zur Sprach zu Text und Text zu Sprach Umwandlung an. Des Weiteren wird ein Service für \ac{nlu} und \ac{nlc}, wobei die Absicht einer Eingabe ermittelt wird, angeboten. Mit Watson Assistant kann ein Chatbot realisiert werden \cite{IBMWatsonSpeechServices}.
	\end{itemize}
	\item \textsl{Lokale Sprachverarbeitung:} Der Einsatz einer lokalen Sprachverarbeitung auf der Laufzeitumgebung bietet einer besser Kontrolle der Privatsphäre
\end{itemize}










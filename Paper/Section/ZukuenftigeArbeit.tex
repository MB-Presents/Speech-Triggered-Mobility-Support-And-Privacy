\section{Zukünftige Arbeit}
Dieser Artikel zeigt das Probleme aktueller Sprachassistenten auf. Mit dem vorgestelltem Konzept kann einem Nutzer eine bessere Privatsphäre gewährleistet werden. Im letzten Kapitel wurden einige Technologien zur Umsetzung dieses Konzeptes vorgestellt. Es zeigte sich, dass es eine große Auswahl an Technologien, gerade im Bereich der Sprachverarbeitung, gibt. Als nächsten Schritt müsste diese Technologien evaluiert werden, um eine Technologieauswahl treffen zu können. Anhand dieser Auswahl kann festgelegt werden, welche Ressourcen die private Cloud zur Nutzung dieses Technologien benötigt. Des Weiteren lässt sich anhand der benötigten Ressourcen sowie Technologien ein genaueres Kostenmodell erstellen. Dieses kann genutzt werden, um eine erneute Umfrage bzw. Interviews mit möglichen Zielgruppen durchzuführen. Die Umfrage soll zeigen, ob die Zielgruppen bereit sind, die zusätzlichen Kosten für eine bessere Privatsphäre zu bezahlen. Danach kann mit einer prototypischen Entwicklung des Sprachassistenten begonnen werden. Zusätzlich gilt es ein Konzept zu entwickeln, mit welchem Apps auf unerwünschten Datenweitergabe an Dritte geprüft werden können. 
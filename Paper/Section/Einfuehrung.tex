\section{Einführung}
Die Sprachsteuerung ist eine Interaktionsmöglichkeit, bei der technische Geräte durch die menschliche Sprache gesteuert werden können. Das nächste Lied, der Wecker oder auch Bestellprozesses können damit initiiert werden.  Die Experten gehen von einem wachsenden Markt für die Sprachsteuerungen aus: Die Fachzeitschrift \glqq PR Newswire\grqq{} vermutet, dass sich Einkäufe über Sprache in den nächsten vier Jahren um das Zwanzigfache ansteigen \cite{prNewswire}. Das Magazin \glqq Campaign\grqq{} schätzt, dass die Suche in Browser über die Tastatur von der Suche über Sprache in Zukunft ersetzt wird \cite{Campaign}. Die Sprachassistenten beinhalten solche Sprachsteuerungsservices und sind somit die Schnittstelle zwischen Nutzer und Anwendung. \newline

Die Serviceleistungen von Apps sind über eine Schnittstelle der Plattform zu erreichen. Die Sprachsteuerung auf der Plattform ist sehr anspruchsvoll, da hier die Spracheingabe viele hochkomplexe Prozessschritte durchläuft. Bisher werden diese Plattformen oftmals nur von großen Konzern angeboten, welche über die finanziellen Mittel und das Knowhow jedes einzelnen Schrittes verfügen. Universitäten konzentrieren sich meist auf einen dieser Schritte. Es verwundert nicht, dass die Sprachassistenten oftmals von Amazon, Google, Microsoft oder Baidu angeboten werden. Die Open-Source-Varianten sind meist in der Funktionalität und Performance limitiert. Bei den kommerziellen Produkten sind Funktionalität und Performance gut, aber dafür gibt es Bedenken hinsichtlich der Privatsphäre. Mobile Geräte, wie Smartphone und Speaker, senden die Spracheingabe für die Auswertung zu den Rechenzentren. Dabei können Daten erfasst und für andere Zwecke verwendet werden. Die Datenverwendung sind in der Nutzungsbestimmung angegeben, allerdings vermitteln diese eine beschränkte Vorstellung von den möglichen Verwendungsszenarien, wie das Profiling. 

Aus diesem Grund haben wir eine Umfrage mit 110 Teilnehmern durchgeführt, welche die Nutzung von Sprachassistenten erfasst, die Relevanz des Datenschutzes aus Nutzersicht ermittelt und dabei die finanzielle Bereitschaft überprüft. Außerdem gaben die Nutzer Anwendungen an, bei denen ihnen ein hoher Datenschutz wichtig ist. Diese Umfrage ist in Abschnitt \ref{umfrage} angeführt. Aus dem Ergebnis lassen sich Schlussfolgerungen ziehen, welche das Konzept für eine Sprachassistentenumgebung zusammengefasst. Das Konzept listet Anforderungen im Hinblick auf eine einschränkungsfreie Plattform. Anschließend wird das Konzept der User-Controlled-Privacy im Konzeptkontext erläutert.
Nachfolgend wird die hybride Architektur beschrieben, welche das Konzept mit Datenschutz und Funktionalität vereinen soll. Es wird auf die Ebenen der Architektur eingegangen, wobei die verschiedenen Komponenten erklärt werden.
Im nächsten Kapitel wird eine Realisierungsmöglichkeit erläutert, bei der die eingeführten Komponenten mit passenden Unterstützungen ergänzt werden. 
Eine Beurteilung des Konzeptes, der Technologien und der Umsetzung schließen den Artikel ab.

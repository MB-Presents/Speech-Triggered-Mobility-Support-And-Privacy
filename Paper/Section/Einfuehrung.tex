\section{Einführung}
Die Sprachsteuerung ist eine Interaktionsmöglichkeit, bei der technische Geräte durch die menschliche Sprache gesteuert werden können. Das nächste Lied, der Wecker oder auch Bestellprozesses können damit initiiert werden.  Die Experten gehen von einem wachsenden Markt für die Sprachsteuerungen aus: Die Fachzeitschrift \glqq PR Newswire\grqq{} vermutet, dass sich Einkäufe über Sprache in den nächsten vier Jahren um das Zwanzigfache ansteigen \cite{prNewswire}. Das Magazin \glqq Campaign\grqq{} schätzt, dass die Suche in Browser über die Tastatur von der Suche über Sprache in Zukunft ersetzt wird \cite{Campaign}. Die Sprachassistenten beinhalten solche Sprachsteuerungsservices und sind somit die Schnittstelle zwischen Nutzer und Anwendung. 

Die Anwendungen eines Sprachassistenten werden Apps genannt und meist auf einer Plattform in der Cloud ausgeführt. Die Sprachverarbeitung auf der Plattform ist anspruchsvoll, da hier die Spracheingabe eines Nutzers hochkomplexe Teilprozesse der Sprachverarbeitung durchläuft, bis eine passende Antwort für den Nutzer erzeugt werden kann. Aktuell werden diese Plattformen von großen Cloud-Anbietern angeboten, welche über die finanziellen Mittel und das Knowhow jedes einzelnen Teilprozesses verfügen. Universitäten konzentrieren sich meist auf einen Teilprozess. Sprachassistenten werden von Amazon, Google, Microsoft oder Baidu angeboten, wobei diese viel Funktionalität und gute Performance bieten. Jedoch gibt es Bedenken hinsichtlich der Privatsphäre, da aus den Datenschutzerklärungen der Cloud-Anbieter nicht klar hervor geht, was mit den Daten eines Nutzer in der Cloud passiert. Mobile Geräte wie Smartphone und Lautsprecher senden die Spracheingabe eines Nutzers für die Auswertung zum entsprechenden Cloud-Anbieter. Dabei können Daten erfasst und für andere Zwecke verwendet werden. Die Datenverwendung sind in der Nutzungsbestimmung angegeben, allerdings vermitteln diese eine beschränkte Vorstellung von den möglichen Verwendungsszenarien, wie das Profiling von Nutzern. 

Aus diesem Grund wurde eine Umfrage mit 110 Teilnehmern durchgeführt, welche die Nutzung von Sprachassistenten erfasst, die Relevanz des Datenschutzes aus Nutzersicht ermittelt und dabei die finanzielle Bereitschaft für mehr Datenschutz überprüft. Außerdem gaben die Nutzer Anwendungen an, bei denen ihnen ein Datenschutz besonders wichtig ist. Diese Ergebnisse werden im Kapitel \ref{sec:motivaiton} erläutert und stellen die Motivation zur Entwicklung eines Konzeptes eines Sprachassistenten mit mehr Privatsphäre für den Nutzer dar. Im Kapitel \ref{sec:konzept} wird ein Konzept vorgestellt durch das ein Nutzer die volle Kontrolle über seine Daten behält. Zur Erfüllung dieses Konzepts wurde eine Architektur für einen Sprachassistenten entwickelt, welche in Kapitel \ref{sec:architecure} vorgestellt wird. Anschließend werden in Kapitel \ref{sec:umsetzung} Technologien vorgestellt, mit welchen die Architektur umgesetzt werden kann. Eine Beurteilung des Konzeptes, der Technologien und der Umsetzung schließen den Artikel ab. \newline

Diese Artikel wurde im Rahmen des Moduls \glqq Mobilität und Innovation\grqq{} an der Fachhochschule Furtwangen und unter Betreuung von Prof. Dr. Achim P. Karduck erstellt. Ein besonderen Dank geht dabei an Prof. Dr. Achim P. Karduck für die gute Betreuung und die vielen Denkanstöße.

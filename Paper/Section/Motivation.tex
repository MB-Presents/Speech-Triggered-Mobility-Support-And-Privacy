\section{Related Work}

Viele Personal Assistant sind auf dem Markt anzutreffen, wie  Alexa, Google Assistant, Siri, Cortana und DuerOS\cite{alexaAssitent} \cite{googleAssistant} \cite{siriAssistent} \cite{cortanaAssistent}\cite{baiduAssistant}. Dabei unterscheiden sich die Sprachassistanten teilweise in der Privatsphäre. 

Amazons Alexa verwendet alle Eingabe, um den Service zu verbessern und personalisierte Werbung anzuzeigen\cite{alexaPrivacy}. Eine Datenfreigabe für verschiedene Bereiche ist möglich, womit sich allerdings auch die Funktionalität einschränkt. 

Google Assistant verwendet die gleichen Berechtigungen, welche für die App gelten\cite{googleShare}. Eine abweichende Einstellung zur App ist dabei nicht möglich. Die Interaktion mit Google Assistant kann für die personalisiert Werbung genutzt werden, wie sonstige Suchanfraugen\cite{googlePrivacy}. 

Bei Siri müssen Dienste aktiviert sein, sodass darauf zugegriffen wird\cite{siriPrivacy}. Um die Aussprache un die Funktionalitäten zu verbessern, werden Daten, wie Name, Kontakte, Musik, Suchaktivitäten und weitere Informationen verschlüsselt übertragen. Die Daten werden nicht mit der Apple ID genutzt, sondern mit zufällig erstellten Kennung mit deinem Geräte. Für die Lernprozesse bleiben die Daten in der Regel auf dem Gerät.

In den Datenschutzeinstellungen von Microsofts Cortana wir darüber informiert, dass bestimmte Daten "[...] wie z. B. Ihre Suchen, Kalender, Kontakte und Orte. [...]"\cite{cortanaAssistent} gespeichert werden. Die Datennutzung von Cortana als Personal Assistant ist konfigurierbar. Sind die personalisierten Informationen deaktiviert, kann Cortana für Aufgaben, wie der Suche, Timereinrichtung und weitere personenunabhängige Funktionen genutzt werden. Cortana verwendet personenbezogene Daten nicht für personalisierte Werbung.  

Baidu sammelt ebenfalls Nutzerdaten, um ihre Dienst zu verbessern. Qi Lu verweist auf die vielen Szenarien in denen Baidu für die Datensammlung eingesetzt wird, womit Baidu den Sprung an die Weltspitze im Bereich Künstliche Intelligenz gelingen soll\cite{baiduAI}. Die persönlichen Daten werden übermittelt. Dabei bietet Baidu keine konfigurierbare Privacy.

Seit Februar gibt es den Mycroft Mark II, womit Offenheit und Privacy eines Smart Speaker vereint werden soll\cite{mycroftsmartspeaker}. Die Funktionalitäten sind hier begrenzt, da der Sprachassistent keine Daten speichert, um den Benutzerkontext weiter zu trainieren und verstehen.